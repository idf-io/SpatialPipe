\chapter*{Abstract}
\addcontentsline{toc}{chapter}{Abstract}


Prior to the emergence of spatial-omics technologies, cell-cell communication (CCC) methods relied on prior knowledge about interacting features between cells. The measure of correlation between putative ligand-receptor interactions are a common approach to this end. Novel spatially-resolved omics technologies like MIBI-TOF and MERFISH enable the measurement and characterization of feature patterns in space. Novel CCC methods followed which attempt to leverage the spatial dimension. These methods face four main challenges: (1) lack of a consensus on how to define and approach CCC, (2) vast diversity of spatial-omics technologies with different output types and properties, (3) high level of artifacts and uncertainty in upstream steps and (4) lack of ground truth for validation.

In the present study, the aim was to set the groundwork towards benchmarking spatially-resolved CCC methods by applying recent spatial CCC methods NCEM and MISTy to a colorectal tumour dataset. We analyzed their potential weaknesses and attempted to reach a comparable output.

The linear NCEM model was applied to the dataset and the interaction term tensor was summarised via two different transformations to achieve a cell-type level sender-receiver effect matrix. Furthermore, erroneous cell-type annotation due to upstream effects was simulated via shuffling the annotations of varying fractions of randomly selected cells. The results showed a high sensitivity towards cell-type annotation as shuffling 10% of the cells' cell-types reached similar explained variance ($R^2$) values to 100% annotation randomization.

MISTy was applied by defining the "intra-view" and a "para-view". As opposed to NCEM, MISTy doesn't model batch and conditions directly; it is run on an individual sample basis and the results are summarized via the mean and standard deviation. When examining the improvement of explained variance ($R^2$) attributed to the "para-view", high standard deviation for all features was observed. This shows the necessity of better sample integration or the assessment of experimental conditions for intermediate step interpretability.

MISTy was customized to yield a similar output to NCEM. However, predictor-target importances were only obtained on the individual sample level.


\pagenumbering{roman}
\setcounter{page}{1}

