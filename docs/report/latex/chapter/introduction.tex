\chapter{Introduction}
\pagenumbering{arabic}
\setcounter{page}{1}



Spatially-resolved omics technologies harness the information found in the spatial dimention to give insight into tissue properties, structure and processes which cannot be fully captured without spatial coordinates. These insights are essential to understanding biological processes like cellular differentiation, immune interactions, tissue homeostasi and disease \cite{Armingol-2021} \cite{Jin-2022}. Cell-cell communication (CCC) underlies many of these biological processes and is one of the desired downstream ouputs of spatial-omics data analysis.

Most cells in multi-cellular organisms take part in cell-cell communication (CCC) \cite{Alberts-2015}, which itself takes place on several levels. Inter-cellular communication can take place via autocrine-, paracrine-, juxtacrine-, endocine-, synaptic signaling and via mechanical forces. Autocrine, juxtacrine, paracrine and endocrine signaling occur via extra-cellular biochemical and protein signaling molecules (ligands), which interact with the receptor proteins of the receiver cells. Receptor proteins can be found on the cell membrane and within the cell (e.g. nuclear envelope), and when activated then trigger downstream intra-cellular signaling cascades mediated via signaling proteins. These systems, or pathways, result in altered effector-protein function. Examples of effector proteins are ion channels, cytoskeleton proteins, metabolite regulators and transcription factors (TF). Alteration of the latter leads to changes in gene expression. Ultimately, altered effector-protein activity changes is the cell's behaviour, which in turn may also lead to further secretion of signaling proteins or other cell-cell interactions (CCI). Autocrine, juxtacrine, paracrine and endocrine communication differ by the distance of influence of the excreted signaling molecules. Autocrine signaling takes place when the sender cell is simultaneously the receiver cell by reacting to its own secreted signaling molecules. Juxtacrine signaling is a contact-dependent signaling process where the signaling molecules are membrane-bound molecules or proteins that bind receptors of immediately neighbouring cells. Paracrine signaling occurs when the released signaling molecules act as local mediators in the local environment. Signaling over longer distances occurs via endocrine signaling, where molecules like hormones travel via specific systems such as blood vessels. Additionally, cells also respond to action potential changes and neurotransmitters (synaptic signaling) and to mechanical forces mediated via transmembrane proteins like integrins. In the context of spatial-omics, CCC is defined as the the effects that arise and are only explainable via spatial relationships between cells and cannot be recovered from individual cell expression profiles.

Empirical measurement of interacting ligand-receptor (L-R) interactions, such as co-immunoprecipitation, fluorescence resonance energy transfer (FRET) imaging
and X-ray crystallography, require high domain knowledge to direct the experiments and are limited by the employed biochemical assays\cite{Armingol-2021}. The main challenge is that the conditions do not take place in the cell's native environment. Transcriptomics-, proteomics- and metabolomics-based computational methods complement the empirical assays and by infering previously unknown interactions. The complexity of coordinated CCC has given rise to a plethora of different CCC approaches (Table \ref{tab:ccc1}, \ref{tab:ccc2}). They can be categorized by two criteria, (1) the use of spatial information and (2) the reliance on prior-knowledge\cite{Almet-2021}\cite{Jin-2022}. 

Transcriptomics-based non-spatial methods mainly rely on knowledge about putative ligand-receptor interactions (LRI), which are found in curated databases. Some database resources, are simplistic and describe binary binary LRIs, while others like CellPhoneDB, CellChat and ICELLNET attempt to reduce false positives by factoring in different protein isoforms and protein-complexes-subunits. It is also common to combine L-R databases to increase the scope of LRIs. Notably, non-spatial and some CCC methods rely on the core assumptions that mRNA expression reliably correlates with protein abundance and that LRIs can be directly infered through their abundance, disregarding effects like post-translational modifications, co-factors and allosteric inhibition.

Most of non-spatial CCC methods fall into four categories (Table \ref{tab:ccc1}, \ref{tab:ccc2})\cite{Armingol-2021}\cite{Almet-2021}\cite{Jin-2022}. The first uses differential expression between annotated clusters to identify interacting L-R pairs. CellTalker, iTalk and PyMINEr are examples of this kind. The second category are network-based methods like NicheNet and SpotSC that take advantage of downstream pathway signaling that is known to be caused by receptor activation to ligand-receptor predict LRIs. The third category computes a LRI score and an additional significance value via cluster permutation or non-parametric tests to establish the null hypothesis in hypothesis testing. Examples of CCC methods implementing the latter approach are CellPhoneDB, CellChat, ICELLNET and Giotto. The fourth kind represents L-R expression across clusters as a rank 3 tensor. scTensor, for example, uses non-negative Tucker decomposition to further decompose this tensor to compute individual LRIs

A subclass of spatial CCC methods follow non-spatial method's footsteps by integrating prior-knowledge into their model. For example, Giotto and CellPhoneDB v3, analogously to permutation based non-spatial methods, calculate the RLIs and use permutation to calculate confidence intervals while also restricting interactions to spatially proximal cells. SpaOTsc generates a network of interacting cells and LRIs by approaching CCC as an optimal transport problem. COMMOT also uses an optimal transport approach that takes into account ligand competition for receptor. Predictive machine learning approaches include NCEM and MISTy. NCEM uses graph neural networks (GNN) to represent spatial location and distances between cells. It then infers node  expression based on the neighbourhood cell-type composition of the index cell. MISTy defines different spatial views and integrates them into a metamodel views to analyze their contribution to variance as well as the measuring the predictor-features importances towards target-feature prediction. Spatial variance decomposition analysis (SVCA) uses probabilistic gaussian processes to disentangle the variation from intra-cellular, environmental and inter-cellular sources on a feature-level. GCNG is an ML model that employs convolutional GNNs to predict cell-type-label agnostic LRIs. Lastly, DIALOGUE takes an unsupervised approach by detecting multi-cellular programs (MCP) based on spatial cross-cell-type variation.

{\scriptsize
    \begin{longtable}[l]{ p{2cm} p{5.5cm} p{5cm} m{2cm} }
        \caption{Current cell-cell communication methods and their main concepts.} \\
        \label{tab:ccc1} \\
        \hline 
        \textbf{Method} & \textbf{Core concept} & \textbf{Output} & \textbf{Programming language} \\ \hline
        SpaOTsc & Optimal transport model which integrates downstream intra-cellular signaling & RLI or pathway scores between individual cells & Python \\[0.7cm]
        CellPhoneDB \newline v3 & LRI score and significance restricted to connected cells in spatial graph & LRI up- or down-regulation and significance for each cluster pair & Python \\[0.7cm]
        Giotto & LRI score (average R-L expression) and significance restricted to connected cells in spatial graph & LRI up- or down-regulation and significance for each cluster pair & Python, R \\[1.2cm]
        stLearn & L-R co-expression score applied on tissue locations with normalized gene expression & Spot-wise LRI across discretized tissue & Python \\[0.7cm]
        MISTy & Non-linear ensemble ML model based on custom spatial relationships & Predictor-target feature relationships & R \\[0.7cm]
        NCEM \newline (linear) & Linear regression based on local neighbourhood annotated cluster presence in spatial graph & Feature-wise interaction terms and significance for each sender-receiver cluster pair & Python \\[1.2cm]
        SVCA & Probabilistic gaussian process model which accounts for intrinsic, environmental and CCI variation & Proportion of variance contributed by the 3 different sources for every gene & Python, R \\[1.2cm]
        DIALOGUE & Unsupervised capture of MCPs based on coordinated gene expression across cell-types and images. Correlation maximization. & Feature up- or down-regulation for each cell type for each MCP & R \\[1.6cm]
        COMMOT & Collective optimal transport which also models ligand-receptor competition & LRI scores for each cell pair & Python \\[0.73cm]
        GCNG & Graph convolutional neural network trained on known LRI & LRI pairs & Python \\[0.7cm]
        iTALK & Differential gene expression (DGE) between clusters, Comparison across conditions & RLI probabilities between cluster pairs & R \\[1.2cm]
        CellTalker & DGE between clusters, Comparison across conditions & Up- or down-regulated gene interactions between cluster pairs & R \\[0.7cm]
        PyMINEr & DGE to detect altered signature pathways & R-L interaction network for each cluster & Python, stand alone app \\[0.7cm]
        NicheNet & LRI scores via L-R downstream signaling integration & L-R interaction scores for each cluster pair & R \\[0.7cm]
        SoptSC & LRI scores via L-R downstream signaling integration & Cell pair or cluster pair CCC probabilities & R, Matlab \\[0.7cm]
        CellPhoneDB \newline v2 & Permutation of cluster annotations for significance calculation using LRI scores (product of expressions) & LRI up- or down-regulation and significance for each cluster pair & Python \\[1.2cm]
        CellChat & LRI scores via Hill equation and permutation to perform hypothesis testing & LRI probability and significance for each cluster pairs & R \\[0.7cm]
        ICELLNET & Sum of LRI scores (product of R-L expressions) & LRI, communication scores and significance for each cluster pair & R \\[0.7cm]
        scTensor & Rank 3 tensor decomposition via non-negative Tucker decomposition & Three matrices with informacion on interacting cells with their LRI & R \\[0.7cm]
        Cell2Cell & L-R pair co-expression & LRI scores for every cluster pair & Python \\[0.4cm]
        \multicolumn{4}{l}{\scriptsize \itshape Y: Yes, N: No, *: Varies under certain choices, parameters or conditions} \\
    \end{longtable}
}




{\scriptsize
    \begin{longtable}[l]{ l l p{2cm} p{2.5cm} p{2cm} l p{2cm} }
        \caption{Current cell-cell communication methods and their main properties.} \\
        \label{tab:ccc2} \\
        \hline
        \textbf{Method} & \textbf{Spatial} & \textbf{Constrained by niche \newline definition} & \textbf{Restricted by prior-information} & \textbf{Annotation required} & \textbf{Predictive} & \textbf{Leave-one-out CV} \\ \hline
        SpaOTsc & Y & N & L-R, pathway & N & Y* & N \\
        CellPhoneDB v3 & Y & N & L-R & Y & N & ~ \\
        Giotto & Y & N & L-R & Y & N & ~ \\
        stLearn & Y & Y & Y & Y & N & ~ \\
        MISTy & Y & N* & N & N & Y & Y \\
        NCEM (linear) & Y & Y & N & Y & Y & N \\
        SVCA & Y & Y & N & N & Y & Y \\
        DIALOGUE & Y & N* & N & Y & N & ~ \\
        COMMOT & Y & Y & R-L & N & N* & N \\
        GCNG & Y & Y & R-L & N & Y* & N \\
        iTALK & N & N & L-R & Y & N & ~ \\
        CellTalker & N & N & L-R & Y & N & ~ \\
        PyMINEr & N & N & L-R, pathway & Y/N & N & ~ \\
        NicheNet & N & N & L-R, pathway & Y & N & ~ \\
        SoptSC & N & N & L-R, pathway & Y & N & ~ \\
        CellPhoneDB v2 & N & N & L-R, multi-subunit & Y & N & ~ \\
        CellChat & N* & N & L-R & Y & N & ~ \\
        ICELLNET & N & N & L-R & Y & N & ~ \\
        scTensor & N & N & N & A & N & ~ \\
        Cell2Cell & N & Y & L-R & Y & N & ~ \\[0.4cm]
        \multicolumn{4}{l}{\scriptsize \itshape Y: Yes, N: No, *: Varies under certain choices, parameters or conditions} \\
    \end{longtable}
}

Spatial CCC methods present their own set of challenges. For example, spatial methods such as NCEM, SVCA and GCNG rely on defining a certain spatial niche based on a scale parameter, e.g., radius (Table \ref{tab:ccc3}). Some challenges are inherited from the non-spatial method analogues when integrating LRIs which contain limited information about true CCC and are oftentimes biased towards the ligands and receptors provided by the knowledge database \cite{Jin-2022}\cite{Dimitrov-2021}. However, CCC problems stem mostly from two sources, (1) the vast diversity in spatial technology data types and (2) the upstream pre-processing steps of this technology output data.

To list a few spatial transcriptomics technologies categorized by their approach \cite{Moses-2022} : Laser capture microdissection (LCM)\cite{Emmert-Buck-1996} and Tomo-seq(Junker-2014) select a region of interest (ROI); seqFISH\cite{Lubeck-2014} and MERFISH\cite{Chen-2015} are based on fluorescence in-situ hybridization (FISH) of mRNA molecules; FISSEQ\cite{Lee-2014} and ExSeq\cite{Alon-2021} use in-situ sequencing (ISS); Slide-seqV2\cite{Marshall-2022}, spatial transcriptomics (ST) \cite{Stahl-2016}, and 10x VISIUM\cite{Moses-2022} and Xenium apply next generation sequencing (NGS) with spatial barcoding on arrays; and DNA microscopy\cite{Weinstein-2019} reconstructs spatial locations based on cDNA proximity. Other omics technologies include spatial proteomics like multiplexed ion beam imaging (MIBI-TOF)\cite{Keren-2019} and seqFISH	extsuperscript{+}\cite{Shah-2018}. There are currently no established upstream protocols or workflows that account for this variation in spatial data types as their preprocessing is mostly adapted to the technologies' specific experimental nature and  intricacies. In addition, tendency shows that different spatial technologies compromise between three areas: multiplexity, detection efficiency and resolution. For example, VISIUM (barcoded array based method), captures high amounts of genes (whole transcriptome) while its resolution remains on a super-cellular level (55 $\mu m$ spots) and its detection efficiency on a moderate scale (~15.5% per area compared with smFISH). ISS-based technology FISSEQ also has transcriptome-wide coverage. FISSEQ's resolution, on the other hand, is sub-cellular while compromising in low detection efficiency (0.005%). Other technologies like smFISH based MERFISH can detect hundred to thousands of genes, show high detection efficiency (~95% for Hamming distance 4) and sub-cellular resolution. 

The second problem are the pre-processing steps which precede CCC. There is currently no consensus on how to approach steps like signal decoding, cell segmentation, quality control and cell-type annotation. These pose significant sources of variation and uncertainty. For example, cell segmentation of spatial data with sub-cellular resolution is influenced by the segmentation method, additional data quality (e.g. cell nucleus staining), capture of mRNA from varying section depths which could belong to a different cell, variety of cell shapes etc. Regarding quality control, there are no clear protocols of batch and condition integration. There is also no consensus on an additional normalization step based on cell size (which is in turn also influenced by segmentation) or accounting for missing feature detection due to low detection efficiency. These are all situations additionally influenced by the output of the spatial-omics technology.

Some CCC methods show potential in overcoming specific uncertainty sources (Table \ref{tab:ccc3}). For example, NCEM only uses cell-types and the connectivity graph as predictors. MISTy and SVCA are cell-type independent and aggregate features across broader spatial areas, possibly reducing segmentation errors and low detection rates. stLearn, on the other hand, normalizes gene expression within broader spatial regions.

{\scriptsize
    \begin{longtable}[l]{ p{1.5cm} p{2cm} p{1.5cm} p{1.5cm} p{1.7cm} p{3cm} p{3cm}}
        \caption{Strengths and weaknesses of CCC in context of spatial-omics upstream uncertainty sources.} \\
        \label{tab:ccc3} \\
        \hline
        \textbf{Method} & \textbf{Overcomes cell segmentation errors?} & \textbf{Requires cluster annotation?} & \textbf{Overcomes cluster annotation?} & \textbf{Accounting for other upstream artifacts and errors?} & \textbf{Pros} & \textbf{Cons} \\[2.3cm] \hline
        SpaOTsc & N & N & Y & N & CCC on a the individual cell location level & Scalability challenge, does not account for ligand competition and choice of K in building the spatial graph (K nearest-neighbours) \\[2.4cm]
        Giotto & N & Y & N & N & LRIs include multi-subunit protein complexes & Limited to spatially adjacent cells, R-L expression and their annotations \\[1.5cm]
        stLearn & Y, by gene normalization within spots & N & Y, cluster densities are calculated per spot & N & Incorporates tissue morphology into expression space & Can dilute information from rare cell-types \\[2.0cm]
        MISTy & Y, by definition of views that summarize cell features over regions & N & Y & N & Flexible, customizable and scalable & Difficult interpretation and hypothesis driven \\[2.0cm]
        NCEM \newline (linear) & N & Y & N & N & Prior-knowledge and leave-one-feature out independent & Low predictive performane \\[1.2cm]
        SVCA & Y, joint modeling of all cells by an aggregator function & N & Y & N & Cell-type independent, models all genes and robust to technical sources of variation including erroneous segmentation & Rigid output giving only broad sample level insight into CCC. Other cell's effect weighted by exponential decay distance. \\ [2.4cm]
        DIALOGUE & Y, feature expression is averaged for each cell-type across within each image & Y & N & Y, low feature detection by average over samples & Unsupervised, averages features over niches/images, no discrete niches. & Averages features over images and is limited by cell and sample number. \\[2.3cm]
        COMMOT & N & N & Y & N & Models L-R competition, to some extent ligand diffusion and functions on a cell-cell interaction basis & Concept not biologically based and difficult to scale \\[2.0cm]
        GCNG & N & N & Y & N & Cell-type independent & Only 3 nearest neighbours considered in the connectivity graph. Trained on L-R prior-knowledge \\[0.4cm] 
        \multicolumn{7}{l}{\scriptsize \itshape Y: Yes, N: No} \\
    \end{longtable}
}

Furthermore, it should be mentioned that many CCC methods can be applied to barcoded array based technologies with super-cell resolution by deconvolving cell-type proportion within the spots with methods like cell2location \cite{Kleshchevnikov-2022}\cite{Moses-2022}.

Going into further detail on two spatial CCC methods, NCEM is a collection of spatial-graph based models that predict gene expression based on neighbourhood label composition\cite{Fischer-2022}. NCEM models are independent of prior-knowledge, false positive prone leave-one-feature-out modeling and aim to include more effects of CCC than L-R signaling. NCEMs are omics independent as their input requires spatial feature expression, single-cell resolution labeling and a radius to define the neighbourhood size. The model with best predictive performance while including interpretable CCI output is the linear model. The linear model predicts the index' cell feature expression via the categorical one-hot encoding of its own label, the presence of neighbourhood cell-types and the batch condition. By modeling  asymmetric combinatorial interaction terms as the neighbourhood composition, linear NCEM yields CCI terms for each sender-receiver label pair. The increase in explained variance ($R^2$) of the model is minimal when compared to the same model without spatial information (baseline model), which the authors attribute to the larger variance explanation due to intra-cell-type variance than inter-cell-type variance. They also claim the model to be robust to segmentation errors, which was simulated by swapping random features between neighbouring nodes. They showed the increase in $R^2$ to be constantly higher in the model with spatial information contrasted again the base model without spatial information. NCEM's predictive performance was also shown to be highly dependent on tissue type, neighbourhood size and spatial omics technology. Non-linear graph neural network based NCEM's showed lower predictive performance than their linear counterpart. Conditional variational autoencoder (CVAE) NCEM modelled cell-intrinsic latent states. This version yielded considerably higher performance (measured in variance explanation) than all other NCEM models but was limited in spatial CCC inference because they were captured in the latent space. Another non-linear NCEM, uses a cell-level L-R activity modeling kernel to predict gene expression. This model showed higher predictive performance than linear, non-linear and CVAE NCEMs. Linear NCEM was additionally shown to be extendable to include spot transcriptomics via cell-type deconvolution method cell2location \cite{Kleshchevnikov-2022}.

MISTy is an explainable machine learning framework which captures spatial relationships from data by defining different spatial and functional views \cite{Tanevski-2022}. Essentially, these views aggregate feature expression in an attempt to describe different cellular contexts like juxtacrine and paracrine signaling. MISTy builds a metamodel which predicts feature expression based on the linear combination of these views and a fixed intra-view. The latter describes the intrinsic variation source and its model is defined by the prediction of individual features within the index cell using the remaining features of the same index cell. MISTy is a non-linear metamodel which is technology and cell-label independent and uses the entire feature space. MISTy takes as input the feature space, spatial data locations, neighbourhood or spatial region defining parameters (depending on the view definition) and an aggregator function. MISTy's high flexibility allows for hypothesis-driven customization which can also includes prior-knowledge integration(e.g. TF-gene network for TF activity inference). MISTy outputs three main result types. The first is measured by contrasting the predictive performance achieved by the model exlusively modeling cell intrinsic variation (intra-view), with the performance achieved by the multi-view model which additionally includes cell-extrinsic variation via custom-views. This metric is described via an increase in variance explained ($R^2$). The second is the contribution of each view to the expression of each marker which is represented by the learnable weight parameters in the metamodel. The third output dissects the importance of each predictor feature from the multi-view in relationship to the predicted target feature by a leave-one-feature-out procedure. Accordingly, the learnable function for each view must be an ensemble algorithm.

Non-spatial CCC methods have proven useful in exploration and validation of cell development, tissue homeostasis and immune interactions. For example, Vento-Tormo et al. applied CellPhoneDB to investigate the maternal–fetal interactions in the decidual-placenta interface. They found trophoblast differentiation-related LRIs like EGFR, NRP2 and MET. Kumar et al. investigated tumour microenvironment (TME) LRIs in mice via L-R expression products. CCR5 interactions and interactions involving the binding to the extra-cellular matrix were found to be positively correlated to tumour growth rate. Another TME study found an association of highly expressed TNS1 fibroblasts with immune excusion via the interaction with cytotoxic T cells in esophageal squamous cell carcinoma \cite{Li-2021}.

The application of spatial CCC methods in publications is increasing ever since their recent emergence. An example of CCC method application outside of the publication research group used the latest version of CellChat which integrates spatial information (> v1.6.0) to analyze immune cell interactions in the spleen as a response to malarial infection\cite{Williams-2023}. CellChat predicted immune interactions between monocytes, B-cell stroma and naïve CD4	extsuperscript{+} T cells to influening the differentiation of the latter. In-house examples include a study into breast cancer which used stLearn to define regions of high cluster heterogeneity and L-R expression\cite{Pham-2020}. They found spots with high interaction probability and high co-expression of CXCR4 and CXCL12, which is consistent with the current literature.  Example signaling interactions were from monocytes to Th1 cells via chemokine and MHCII via Ccl4, Cxcl9, and Cxcl10. Tanevski et al. identified the increasing loss of tissue structure in breast cancer. They analyzed tumour grades one to three and identified a decline of important interactions from the paraview with emphasis on alveoli structural proteins cytokeratin 7 and smooth muscle actin.

Finally, it is worth noting that validation and benchmarking of CCC methods is hindered by the lack of ground truth and the limited pior-knowledge validated experimentally\cite{Almet-2021}\cite{Dimitrov-2021}. Current experimental approaches rely on parallel in-vitro or in-vivo experiments\cite{Sheikh-2019} such as western blot or immunohisto-chemistry, and in-silico simulations to provide simulated ground truth\cite{Tanevski-2022}. Computational benchmarking approaches of non-spatial CCC methods have relied on the comparison of overlapping highest ranked interactions between methods applied using the same prior-knowledge resource. High dissimilarities in the latter aspect and biases of prior-knowledge resources towards specific biological pathways and interactions are issues expected to be transferable to prior-knowledge-based spatial CCC methods.

The aim of the present study is to apply recent spatial CCC methods NCEM and MISTy, which show potential to overcome specific upstream uncertainty sources. Their strengths and weaknesses are identified and the output is evaluated for benchmarking purposes. A proteomics MIBI-TOF dataset of colorectal carcinoma tissue was selected \cite{Hartmann-2021}. Exploratory data analysis was performed to investigate cell-type frequency distribution, recover the major cell lineage profiles of the original publication and to analyze the sources of variation. Spatial statistics were applied to examine pre-defined cell-type annotation distribution across space and to identify spatially variable features. Furthermore, NCEM and MISTy were applied and modified to attempt to achieve a meaningful comparable output. Cell-type annotation sensitivity of NCEM was tested via cell-type shuffling. The results of the study lay the ground-work to CCC method benchmarking.