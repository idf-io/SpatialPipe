\chapter{Discussion}
	


In the present study spatial cell-cell communication methods NCEM and MISTy were applied and evaluated for benchmarking on the \cite{Hartmann-2021} colorectal cancer MIBI-TOF dataset. Applying the linear NCEM model output the interaction terms coefficients, also refered to as sender-receiver effects. These were further summarized via two approaches: (1) quantifying the amount of significant features per sender-receiver cell-type pair and (2) calculating the L2-norm of interaction terms matrix which had previously been filtered by a significance threshold. There were some distinct interactions found in both summarizing statistics, for example the interaction between sender epithelial and receiver CD4\textsuperscript{+} T cells. However, there were also dissimilarities present as the interaction between epithelial and CD11c myeloid cells was not as high in approach 1 compared to approach 2, relative to all interactino values. Also, in the first approach, other CD45\textsuperscript{+} immune cells showed high interactions with most cell-types both as sender and receiver whilst in approach 2, most of these interactions were diluted. None of these interactions can be validated due to the lack of ground truth. Common upstream errors and artifacts are one of the major challenges in the spatial-omics field. For example, incorrect feature detection attribution to its corresponding cell due to imprecise cell-segmentation can lead to incorrect cell-type labeling. We analyzed the sensitivity of the linear NCEM model towards cell-type labels by shuffling the cell-type of varying fractions of randomly selected cells. The high sensitivity even at low shuffling fractions like 10\%, which is common in spatial-omics processing, indicates that variation in the pre-processing steps will have a great effect on linear NCEM output and that both the spatial technology used to obtain the data and the pre-processing steps effectuated before NCEM application must be assessed carefully to account for false annotation.

The flexibility of MISTy to define the views and independence of cell-type annotation was the key reason for its application. Additionally, defining the "para-view" as the weighted sum of the neighbourhood features promised more robustness from upstream artifacts and segmentation errors. However, a different set of problems were encountered upon application compared to NCEM. First, MISTy natively runs on on individual samples and aggregates their results via summary statistics like the mean and standard deviation. The standard deviation obtained for the increase of explained variance ($R^2$) was very high and virtually always in the range of the own mean value. Also, the top ranked features for this result varied greatly when inspecting individual samples (e.g. colorectal carcinoma and healthy colon). The metamodel learnable parameters for each view, also referenced to as contributions, showed continuously higher contributions from the "intra-view" than the "para-view" with exceptions in the individual sample resolution. The higher HIFA and Ki67 contributions from the "para-view" effect found in the colorectal carcinoma sample "Point 23" which were not found in the healthy colon sample "Point49" and the aggregated results could be a hint towards the effect of the cancer. However, making a statement would be speculative due to the lack of ground truth. The feature importances results calculated as the difference between the "juxta-view" and the "intra-view" showed CK and CD11C to be relevant predictors.  with ASCT2, ATP5A, E-cadherin, CPT1A, CS, SDHA, SMA, VDAC1 and S6p. Target features E-cadherin and citrate synthase were specific to the colorectal carcinoma sample. E-cadherin has a key role in tumour metastasis \cite{Beavon-2000}. Disregulation of citrate synthase has been found to be linked to tumour malignancy and the Warburg effect\cite{Lin-2012}. The discrepancy between results of individual images and the aggregated result was observed with features like CPT1A , which was specific to the aggregated result, and S6p and VDAC1, which were high in the colorectal cancer and healthy samples but not in the aggregated result.

The second challenge was the lack of comparable output with linear NCEM's output. To this aim, the MISTy workflow was adapted to use the one-hot encoding of the index cell-types as input and the intra-view was bypassed. As such, the model would predict the index cell-type based on the count of neighbouring cell-types. The result type with comparable format to NCEM were the importances between the cell-type predictor-target pairs. For the aggregated results, no importances were found, while the individual examples showed that endothelial was a predominant predictor in the colorectal carcioma sample and CD4\textsuperscript{+} T-cells and CD11c\textsuperscript{+} myeloid cells were predominant in the healthy colon sample.

The application of MISTy showed the need for a better way to integrate different samples, donors and conditions. Alternatively, the user must have a precise comprehension about the biology and experimental setup to apply MISTy in a tailored manner and make any statements based on MISTy's output. MISTy's flexibility allowed for shaping of the output to our specific purposes. This came at the cost of low interpretability. Again, validation of the results was not possible due to the lack of ground truth.

In contrast to MISTy, NCEM models the sample or batch directly into the linear model. This could be practical to integrate samples. However, accounting for conditions in this way could influence the interaction terms and the final output. Similarly to MISTy, it is evident that the user must have a good understanding of the experimental setup to judge if all samples and conditions are to be included in the application of linear NCEM on a colletive dataset.

In order to critically assess these methods, the dataset was also statistically characterized on the spatial and single-cell level. Cell-type frequency distribution showed a high variability in cell-type frequencies across all samples and also between conditions which may lead to unbalanced model training in NCEM and MISTy, especially the latter since it is applied on an individual sample basis. By summarizing the mean feature expression of cell-types across all samples, the major immune lineage profiles were recovered based on specific markers. PCA analysis, clustering and UMAP embedding did not separate these cell-types clearly, indicating possible upstream annotations errors or the need for better sample integration. Spatial statistics verified the clustering of epithelial cells and immune cells as per colon tissue structure. In addition, Moran's I autocorrelation analysis uncovered features with clustered expression distribution which differed from the lineage markers: GLUT1, HK1, pyruvate kinase M2 (PKM2) and lactate dehydrogenase A (LDHA) in the colorectal carcinoma sample "Point23". GLUT1, PKM2 and LDHA were spatially enriched in most cells except cancerous epithelial cells. They are key proteins in channels and enzymes in the energy metabolism, possibly indicating the activation of immune cells near to the tumour-immune border.

In the big picture, the study highlights the lack of a consensus on the approach to quantify cell-cell communication effects which needs urgent reevaluation to adapt to the latest approaches and technologies. Dependence of spatial CCC methods from upstream steps and artifacts was also a repeated challenge throughout the study which was shown to contribute to unaccounted variation. The high number of uncertainty sources leads to the necessity of the establishment of new protocols, redefinition of concepts  and the development of new methods that take the latter factors into account. The lack of ground truth will also continue to obstruct spatial single-cell benchmarking and evaluation.

To continue the investigation, the outlook includes the application of other spatial CCC methods with different approaches. Non-predictive approaches like DIALOGUE, COMMOT and possibly other non-spatial methods (e.g. CellChat) could leverage other cell properties. Methods that integrate prior-knowledge like ligand-receptor interactions have great potential (e.g. CellPhoneDB v3 or Giotto).

The optimal approach to CCC inference is unclear. Ligand-receptor and protein-protein interactions are essential to cell-cell communication. However, their complexity is not captured well by current spatial technology data. Spatial-transcriptomics measures mRNA, which is only a proxy for protein abundance. Spatial proteomics technologies have a lower-plexity and don't capture the entire landscape of post-translational modifications. Relevant dimentions which are only partially accounted for in current CCC methods are protein-complex subunits, ligand-receptor competition, allosteric regulation and distal endocrine signalling amongst others. On the other hand, data-driven approaches offer prior-knowledge agnostic insights. These in turn are difficult to interpret and their source can be source  can be unclear or less relevant to CCC. Hence, the best CCC approach could be a combination of both approaches to leverage the known effects behind CCC while incorporating other effects agnostically extracted from the data. In addition, CCC will substantially benefit of improving spatial-omics technologies.

Furthermore, the comparison and benchmarking of CCC methods and the effect of upstream steps should remain the focus of further efforts. In-silico simulated spatial data could be used to benchmark the methods. CCC methods could also be applied to other datasets from a diverse range of technologies and omics to evaluate the robustness and flexibility of the CCC methods towards different technologies and omics. Applying NCEM and MISTy on the PCA space could be analyzed for predictive performance improvement. 



\section*{Code availability}

All the analysis, code, conda environment data, figures, input and output data are publically available at \url{https://github.com/idf-io/SpatialPipe}.



\section*{Data availability}

The colorectal carcinoma MIBI-TOF dataset (Methods) is publicably available at \url{https://zenodo.org/record/3951613}.
